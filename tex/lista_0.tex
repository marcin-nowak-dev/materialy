\documentclass[12pt]{article}
\usepackage[utf8]{inputenc}
\usepackage{polski}
\usepackage[a4paper, left=2.0cm, right=2.0cm, top=2.0cm, bottom=2.0cm]{geometry}
\usepackage{hyperref}

\title{PIISW, W08, IO, 2020/2021, semestr letni\\Lista zadań nr 0}
\author{mgr inż. Maciej Małecki\\\small{maciej.malecki@pwr.edu.pl}}

\begin{document}
    \maketitle

    \begin{enumerate}
        \item Następujące narzędzia powinny być zainstalowane na lokalnej stacji roboczej:
            \begin{enumerate}
                \item JDK~8 lub nowsza (zmienna \texttt{JAVA\_HOME} musi być prawidłowo ustawiona w systemie).
                \item GIT (ewentualnie klient graficzny, np GitExtensions)
                \item IDE (sugerowane - alternatywy):
                    \begin{enumerate}
                        \item Idea IntelliJ Ultimate (frontend + backend)
                        \item Visual Studio Code (frontend + backend)
                        \item Eclipse (tylko backend)
                    \end{enumerate}
            \end{enumerate}
            
            W~przypadku IntelliJ Ultimate - jest to narzędzie komercyjne (płatne), jednakże dla studentów istnieje możliwość uzyskania darmowej licencji na tę wersję. W~celu uzyskania takiej licencji należy zarejestrować się korzystając z~uczelnianego konta pocztowego (w~domenie \texttt{pwr.edu.pl}).

            Klienta GIT można pobrać ze strony~\url{https://git-scm.com/downloads}. Klienta GIT Extensions (tylko dla systemu Windows\footnote{Dla systemu Linux/iOS konieczne jest użycie biblioteki Mono.}) można pobrać ze strony~\url{http://gitextensions.github.io/}. \textbf{Uwaga:} korzystanie z~narzędzia GIT Extensions nie jest konieczne do zaliczenia zajęć.

        \item Do realizacji projektu niezbędny jest dostęp do portalu GitHub (\url{https://github.com}). Każdy student powinien utworzyć swoje prywatne konto na tym portalu.

        \item Należy odpowiednio skonfigurować Git’a tak, aby dla commitera używany był adres e-mail użyty do założenia konta na \texttt{github.com}.

%        \item Każdy ze studentów, korzystając ze swojego konta studenckiego, powinien dołączyć do organizacji \texttt{pwr-piisw}: \url{https://github.com/pwr-piisw}. W~tym celu należy podać identyfikator konta prowadzącemu.

        \item Każdy student powinien utworzyć konto na portalu CircleCI (\url{https://circleci.com}). Należy użyć GitHub jako metody autentykacji.

        \item Prosimy o dołączanie do czatów na \url{https://gitter.im} odpowiednich dla każdej grupy laboratoryjnej.
    \end{enumerate}
\end{document}

