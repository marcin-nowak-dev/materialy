\documentclass[12pt]{article}
\usepackage[utf8]{inputenc}
\usepackage{polski}
\usepackage[a4paper, left=2.0cm, right=2.0cm, top=2.0cm, bottom=2.0cm]{geometry}
\usepackage{hyperref}
\usepackage{skull}

\title{PIISW, W-4, IO, 2021/2022, semestr letni\\Lista zadań nr 5: Angular}
\author{Maciej Małecki\\ \small maciej.malecki@pwr.edu.pl}

\begin{document}
    \maketitle

    \section*{Wprowadzenie}
        \begin{enumerate}
            \item W~celu realizacji listy należy założyć nowe prywatne repozytorium oraz zainicjować je zawartością z~\href{https://github.com/pwr-piisw/bookstore}{pwr-piisw/bookstore}.
			\item Repozytorium musi być zintegrowane z~CIRCLE~CI, w~momencie oddawania list status CI powinien być \textbf{zielony}.
			\item Plik \texttt{README.md} zawiera wszelkie informacje potrzebne do zbudowania i~uruchomienia aplikacji.
        \end{enumerate}

    \section*{Oceny}
    \begin{tabular}{|l|c|c|c|c|c|c|c|}
        \hline
        Punkty: & $<8$ & $8-9$ & $10-11$ & $12$ & $13-14$ & $15-16$ & $>16$ \\
        \hline
        Ocena:  & $2,0$ & $3,0$ & $3,5$ & $4,0$ & $4,5$ & $5,0$ & $5,5$ \\
        \hline
    \end{tabular}

    \section*{Zadania}
    \begin{enumerate}

		\item
			(4 pkt) W~aplikacji Bookstore zaimplementowano widok listy książek (dostępny pod linkiem \texttt{ui/books}). Zaimplementuj widok szczegółowy dla książki.
			\begin{enumerate}
				\item W~serwisie \texttt{BooksRestService} zaimplementuj metodę \texttt{findBookById} i~zintegruj ją z~odpowiednim endpointem w~backendzie.
				\item Wygeneruj (przy użyciu Angular CLI) komponent \texttt{book-details} w~module \texttt{books}.
				\item Dodaj routing \texttt{ui/books/\{bookId\}}, dane do widoku \texttt{book-details} wczytaj przy pomocy resolvera.
				\item Dodaj nawigację między widokiem listy oraz szczegółów przy pomocy linków (strona nie może się przeładowywać).
			\end{enumerate}

		\item
			(4 pkt) Rozszerz widok szczegółowy book-details o~listę recenzji. Wyświetl wszystkie recenzje jedna pod drugą. Każda recenzja powinna zawierać autora, tytuł, treść oraz ocenę.
			\begin{enumerate}
				\item Wyświetlanie recenzji powinno być zrealizowane przy użyciu osobnego komponentu (jeden komponent na recenzję, użyj \texttt{*ngFor} do wyświetlenia listy).
				\item Zaimplementuj osobny serwis do wczytywania recenzji.
				\item Zintegruj wczytywanie recenzji z~widokiem widoku szczegółowego.
				\item $\skull$(dodatkowo 2 pkt) Wczytywanie recenzji zrealizuj niezależnie od resolvera dla widoku \texttt{book-details}. Wyświetl ikonę spinnera na czas wyświetlania. Dodaj sztucznie opóźnienie do backendu, aby spinner wyświetlił się przez chwilę.
			\end{enumerate}

		\item
			(4 pkt) Dodaj możliwość dodawania nowej recenzji.
			\begin{enumerate}
				\item Stwórz komponent edycyjny z~polami formularza pozwalającymi na podanie autora, tytułu i~treści. Komponent edycyjny można zrealizować zarówno jako osobny widok lub też pokazać go na widoku \texttt{book-details}.
				\item Wszystkie pola powinny być obowiązkowe - zaimplementuj odpowiednią walidację.
				\item Rozszerz serwis dla recenzji o funkcjonalność zapisu recenzji.
				\item Po zatwierdzeniu recenzji (przy użyciu np. przycisku ``Save'') recenzja powinna być zapisana w~backendzie, a~widok szczegółowy odpowiednio zaktualizowany.
			\end{enumerate}

		\item
			(4 pkt) Zaimplementuj usuwanie książki.
			\begin{enumerate}
				\item Usunięcie książki powinno nastąpić po kliknięciu w~dedykowany przycisk na widoku szczegółowym.
				\item Wyświetl modalne okienko z~potwierdzeniem, zanim usuniesz książkę.
				\item $\skull$(dodatkowo 2 pkt) Usunięcie książki jest możliwe tylko wtedy, gdy nie istnieje żadna recenzja dla danej książki. Dokonaj odpowiedniego sprawdzenia przy pomocy osobnego zapytania RESTowego (poprzez serwis).
				\item Po usunięciu książki aplikacja powinna przełączyć się na widok listy książek.
			\end{enumerate}

    \end{enumerate}
\end{document}

