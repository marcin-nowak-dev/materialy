\documentclass[12pt]{article}
\usepackage[utf8]{inputenc}
\usepackage{polski}
\usepackage[a4paper, left=2.0cm, right=2.0cm, top=2.0cm, bottom=2.0cm]{geometry}
\usepackage{hyperref}
\usepackage{skull}

\title{PIISW, W08, IO, 2018/2019, semestr letni\\Lista zadań nr 5}
\author{Maciej Małecki\\ \small maciej.malecki@pwr.edu.pl}

\begin{document}
    \maketitle

    \section*{Wprowadzenie}
        \begin{enumerate}
            \item Zadania z~poniższej listy należy zrealizować w~ramach prywatnego repozytorium założonego dla listy nr.~1.
            \item Projekt z~repozytorium należy co najmniej raz zbudować wykorzystując do tego komendę \texttt{mvnw install}. Pozwoli to na zainstalowanie niezbędnych narzędzi takich jak \texttt{node}, \texttt{npm}, \texttt{yarn} oraz \texttt{Angular CLI} oraz zbudowanie części \textit{backendowej} aplikacji.
            \item
                Do pracy z~\textit{frontendem} konieczne jest uruchomienie części \textit{backendowej}:
                \begin{verbatim}
java -jar target/ng-boot-project-seed-0.0.1-SNAPSHOT.jar
                \end{verbatim}
                Jeśli z~jakiegoś powodu domyślny port \texttt{8080} jest zajęty, można zmienić tę wartość (np. na~\texttt{8081}) modyfikując plik \texttt{application.properties}.
            \item Interakcja z Angular CLI wymaga użycia konsoli/terminala. W~ramach terminala należy przejść do katalogu głównego sklonowanego repozytorium oraz uruchomić skrypt \texttt{front\_cli.bat}. Użytkownicy systemów Linux lub OS~X powinni wykonać komendę: \texttt{source front\_\allowbreak cli}.
                Wszelkie komendy \texttt{Angular CLI} należy wydawać będąc w~podkatalogu \texttt{src/main/client}.
            \item Wszystkie zadania powinny być zrealizowane w~ramach prywatnego repozytorium na GitHub. Repozytorium musi być zintegrowane z~Travis~CI, w~momencie oddawania list status CI powinien być \textbf{zielony}.
            \item Należy dołożyć wszelkich starań, aby tworzona aplikacja była właściwie przetestowana przy pomocy testów jednostkowych (zarówno strona \textit{backend} jak i~\textit{frontend}.
            \item Przy realizacji zadań pomocne będą materiały z~wykładu (\href{https://pwr-piisw.github.io/wyklady/angular_1.html#/}{angular}, \href{https://pwr-piisw.github.io/wyklady/reactive.html#/}{programowanie reaktywne}) oraz przykładowe repozytorium (\href{https://github.com/pwr-piisw/angular-wyklad}{angular-wykład}) - warto przeanalizować całą historię zmian.
            \item Ilość wszystkich punktów możliwych do uzyskania z~tej listy jest większa niż jest to konieczne do uzyskania oceny $5,0$. Zadania lub punkty oznaczone symbolem $\skull$ należy potraktować jako w~całości lub w~części nieobowiązkowe (odpowiednio). Zadania \ref{exc:yarn} oraz \ref{exc:backend} należy zrealizować co najmniej w~takim stopniu, który umożliwi realizację innych zadań (zadania nie są niezależne od siebie).
        \end{enumerate}

    \section*{Oceny}
    \begin{tabular}{|l|c|c|c|c|c|c|c|}
        \hline
        Punkty: & $<8$ & $8-9$ & $10$ & $11-12$ & $13$ & $14-15$ & $>15$ \\
        \hline
        Ocena:  & $2,0$ & $3,0$ & $3,5$ & $4,0$ & $4,5$ & $5,0$ & $5,5$ \\
        \hline
    \end{tabular}

    \section*{Zadania}
    \begin{enumerate}
        \item\label{exc:yarn}
            (4 pkt) Zarządzanie zależnościami przy użyciu \texttt{yarn}.
            \begin{enumerate}
                \item Przeanalizuj zawartość pliku \texttt{src/main/client/package.json}, sekcje: \texttt{de\-pen\-den\-cies} oraz \texttt{devDe\-pen\-den\-cies} a~także pliku \texttt{yarn.lock}. Korzystając ze strony \url{https://semver.npmjs.com/} sprawdź, jak interpretowany jest numer wersji rozpoczynający się od znaku \texttt{\textasciitilde} oraz od \texttt{\textasciicircum}, a~jak interpretowany jest numer wersji podany ,,wprost''.
                \item Korzystając z~odpowiedniej formy komendy \texttt{yarn} zmień wszystkie wersje referowane przy pomocy \texttt{\textasciicircum} na takie, które są referowane przez \texttt{\textasciitilde}.
                \item Korzystając z~odpowiedniej formy komendy \texttt{yarn} zmigruj projekt do Angulara 5.2.0 określając przy okazji, że projekt powinien automatycznie otrzymywać poprawki błędów dla tej biblioteki\footnote{Zwróć uwagę, że Angular to tak na prawdę kilka/kilkanaście bibliotek referowanych niezależnie}. Korzystając z~komendy \texttt{git} przeanalizuj zmiany, jakie pojawiły się w~plikach \texttt{package.json} oraz \texttt{yarn.lock}. Czy potrafisz określić znaczenie pliku \texttt{yarn.lock}?
				\item $\skull$ (dodatkowe 2 pkt) W~analogiczny sposób zmigruj projekt do Angulara 7.x.x.
                \item Po zmianach aplikacja powinna być przetestowana lokalnie oraz na \texttt{travis-ci}.
            \end{enumerate}

        \item\label{exc:backend}
            (5 pkt) Twórca aplikacji \textit{Bookstore} zawartej w~repozytorium \href{https://github.com/pwr-piisw/oasp4js-ng-boot-project-seed}{oasp4js-ng-boot-project-seed} nie dokończył integracji z~backendowym serwisem \texttt{Book\-Rest}\footnote{Zobacz: \texttt{src/main/java/com/capgemini/books/rest/BookRest.java}}. Większość integrakcji z~częścią serwerową jest zasymulowana całkowicie po stronie klienta.
            \begin{enumerate}
                \item Aktualna implementacja serwisu frontendowego (\texttt{book.service.ts}) korzysta z~przestarzałego API klienta Http. Użyj nowego API opublikowanego w~module \texttt{@angular/\-common/\-http}. Pamiętaj o~refactoringu testów (\texttt{*.spec.ts}).
                \item Zintegruj istniejący interfejs użytkownika dla komponentów \texttt{book-\allowbreak de\-tails} oraz \texttt{book-\allowbreak over-\allowbreak view} w~ten sposób, aby użyte były operacje REST dla wczytania listy książek, wczytania pojedynczej książki, dodania nowej książki oraz aktualizacji istniejącej książki.
                \item Dodaj do interfejsu użytkownika możliwość usuwania wybranej książki.
            \end{enumerate}

        \item\label{exc:module}
            (5 pkt) Stwórz nową funkcjonalność w~aplikacji \textit{Bookstore} który pozwala na zarządzanie użytkownikami. Napisz testy jednostkowe.
            \begin{enumerate}
                \item Funkcjonalność powinna być zrealizowana jako nowy moduł (\textit{ng-module}).
                \item Funkcjonalność powinna składać się z~dwóch widoków: listy użytkowników oraz edytora szczegółów użytkownika.
                \item Funkcjonalność powinna realizować operacje \texttt{findAll}, \texttt{save} oraz \texttt{delete}.
                \item Funkcjonalność może być zamockowana całkowicie po stronie frontendu (brak implementacji serwisu REST).
                \item $\skull$ (dodatkowe 2~pkt) : stwórz część backendową dla użytkowników (na poziomie analogicznym do modułu książek) oraz zintegruj ją z~frontendem.
            \end{enumerate}
        \item\label{exc:component}
            (6 pkt) $\skull$ Napisz uniwersalny komponent listy elementów.
            \begin{enumerate}
              \item Komponent powinien umożliwiać pracę z~dowolnymi tablicami elementów (w~tym tablicami obiektów).
              \item Komponent powinien umożliwiać zdefiniowanie nagłówka listy (nazw każdej z kolumn).
              \item Po najechaniu kursorem myszy na dowolny wiersz nie będący nagłówkiem, cały wiersz powinien zostać wyróżniony innym kolorem (podświetlony).
              \item Po kliknięciu na dowolnym wierszu nie będącym nagłówkiem komponent powinien generować wstawiać wartość wybranego obiektu do asynchronicznego strumienia danych (\textit{Observable}).
              \item Użyj tego komponentu w~istniejących widokach listy użytkowników oraz listy książek oraz zintegruj z~istniejącą funkcjonalnością. Zaktualizuj odpowiednio testy jednostkowe.
            \end{enumerate}
    \end{enumerate}
\end{document}

